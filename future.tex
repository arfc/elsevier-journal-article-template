\section{Future Nuclear Projections}
This section lists the future projections and plans of
different EU countries. Only the major contributors have been listed.

\subsection{France}
Energy Transition for Green Growth Bill passed by the National Assembly for 50\% of
nuclear contribution for electricity by 2025, from 63.2GWe. The 2015 amendment tried to remove
the cap, but this was not accepted by the lower house. Currently there is a
1600MWe reactor unit under construction at Flamanville, for operation from 2018.
However, the general atmosphere is to close nuclear power plants when they reach the end
of their lifetime, while not building new ones. This is due to the decreasing
public support for nuclear, along with the \gls{EDF}'s troubling financial situation.

\subsection{Germany}
Germany, despite its past success with nuclear energy back in the 1970s,
has a dwindling nuclear program. The two incidents, Chernobyl and Fukushima,
caused the public and the government to stall and reduce its nuclear fleet.
In March 2011, the government declared a moratorium,which planned to immediately
shut down nuclear power plants that began prior to 1980. Furthermore, the government
passed a phase-out plan to close all reactors by 2022.

\subsection{Czech Republic}
The Czech republic currently has six reactor units, all \gls{VVER}s, which
has licenses that expires in the 2020s. The Czech nuclear program is 
relatively optimistic, with four reactors planning to be constructed in the
next 25 years.

\subsection{United Kingdom}
United Kingdom not only has a substantial amount of nuclear power (total of 15 units
and $8,883$MWe), but also has fuel cycle facilities like a reprocessing plant. The government
claimed that nearly 25GWe of nuclear should be generated by 2025, but there hasn't been
a fixed amount. After the government became more favorable towards nuclear in 2006, utilities
began planning for new plants. Also, many different foreign utilities, from France (EDF),
Russia (Rosatom), and China (China General Nuclear Group). This led up to a total proposed
number of 13 units ($17,900$ MWe) by 2030, with a mixture of EPR, ABWR, and AP1000s.


\subsection{Belgium}
Belgium currently has seven nuclear reactors ($5,943$ MWe), but with little government support
for nuclear energy, all of them are expected to shutdown in 2025. The government
said that its stance on nuclear phase-out "is now final". There are no known plans
to build new reactors.

\subsection{Czech Republic}
The Czech Republic has six nuclear reactors($3,904$ MWe), and the government is committed to the
future of nuclear energy, which is reflected in its 2015 energy policy to build 
more nuclear power plants. The country plans to build two more ($2,400$ MWe) AP1000s by 2035.
The public support of nuclear energy is also very strong.

\subsection{Sweden}
Sweden has nine reactors that generate 40\% of the country's electricity. 
Sweden had four reactors that would close by 2020 due to decreasing profits,
with the nuclear tax of .75 Euro cents/kWh, which is one-third of the operating
cost. The country plans to shut down all reactors by 2050.


with all the other countries only taking up a minute portion of the entire
EU nuclear fleet, it is assumed that they will remain the same.
From this scenario the following assumptions are made
[specific plan for EU nuclear]

This benchmark is then modified depending on the French policy decisions.