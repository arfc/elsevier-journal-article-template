%        File: revise.tex
%     Created: Wed Oct 27 02:00 PM 2018 P
% Last Change: Wed Oct 27 02:00 PM 2018 P
%

%
% Copyright 2007, 2008, 2009 Elsevier Ltd
%
% This file is part of the 'Elsarticle Bundle'.
% ---------------------------------------------
%
% It may be distributed under the conditions of the LaTeX Project Public
% License, either version 1.2 of this license or (at your option) any
% later version.  The latest version of this license is in
%    http://www.latex-project.org/lppl.txt
% and version 1.2 or later is part of all distributions of LaTeX
% version 1999/12/01 or later.
%
% The list of all files belonging to the 'Elsarticle Bundle' is
% given in the file `manifest.txt'.
%

% Template article for Elsevier's document class `elsarticle'
% with numbered style bibliographic references
% SP 2008/03/01
%
%
%
% $Id: elsarticle-template-num.tex 4 2009-10-24 08:22:58Z rishi $
%
%
%\documentclass[preprint,12pt]{elsarticle}
\documentclass[answers,11pt]{exam}

% \documentclass[preprint,review,12pt]{elsarticle}

% Use the options 1p,twocolumn; 3p; 3p,twocolumn; 5p; or 5p,twocolumn
% for a journal layout:
% \documentclass[final,1p,times]{elsarticle}
% \documentclass[final,1p,times,twocolumn]{elsarticle}
% \documentclass[final,3p,times]{elsarticle}
% \documentclass[final,3p,times,twocolumn]{elsarticle}
% \documentclass[final,5p,times]{elsarticle}
% \documentclass[final,5p,times,twocolumn]{elsarticle}

% if you use PostScript figures in your article
% use the graphics package for simple commands
% \usepackage{graphics}
% or use the graphicx package for more complicated commands
\usepackage{graphicx}
% or use the epsfig package if you prefer to use the old commands
% \usepackage{epsfig}

% The amssymb package provides various useful mathematical symbols
\usepackage{amssymb}
% The amsthm package provides extended theorem environments
% \usepackage{amsthm}
\usepackage{amsmath}

% The lineno packages adds line numbers. Start line numbering with
% \begin{linenumbers}, end it with \end{linenumbers}. Or switch it on
% for the whole article with \linenumbers after \end{frontmatter}.
\usepackage{lineno}

% I like to be in control
\usepackage{placeins}

% natbib.sty is loaded by default. However, natbib options can be
% provided with \biboptions{...} command. Following options are
% valid:

%   round  -  round parentheses are used (default)
%   square -  square brackets are used   [option]
%   curly  -  curly braces are used      {option}
%   angle  -  angle brackets are used    <option>
%   semicolon  -  multiple citations separated by semi-colon
%   colon  - same as semicolon, an earlier confusion
%   comma  -  separated by comma
%   numbers-  selects numerical citations
%   super  -  numerical citations as superscripts
%   sort   -  sorts multiple citations according to order in ref. list
%   sort&compress   -  like sort, but also compresses numerical citations
%   compress - compresses without sorting
%
% \biboptions{comma,round}

% \biboptions{}


% Katy Huff addtions
\usepackage{xspace}
\usepackage{color}

\usepackage{multirow}
\usepackage[hyphens]{url}


\usepackage[acronym,toc]{glossaries}
\include{acros}

\makeglossaries

%\journal{Annals of Nuclear Energy}

\begin{document}

%\begin{frontmatter}

% Title, authors and addresses

% use the tnoteref command within \title for footnotes;
% use the tnotetext command for the associated footnote;
% use the fnref command within \author or \address for footnotes;
% use the fntext command for the associated footnote;
% use the corref command within \author for corresponding author footnotes;
% use the cortext command for the associated footnote;
% use the ead command for the email address,
% and the form \ead[url] for the home page:
%
% \title{Title\tnoteref{label1}}
% \tnotetext[label1]{}
% \author{Name\corref{cor1}\fnref{label2}}
% \ead{email address}
% \ead[url]{home page}
% \fntext[label2]{}
% \cortext[cor1]{}
% \address{Address\fnref{label3}}
% \fntext[label3]{}

\title{Synergistic Spent Nuclear Fuel Dynamics Within the European Union\\
        \large Response to Review Comments}
\author{Jin Whan Bae, Clifford E. Singer, Kathryn D. Huff}

% use optional labels to link authors explicitly to addresses:
% \author[label1,label2]{<author name>}
% \address[label1]{<address>}
% \address[label2]{<address>}


%\author[uiuc]{Kathryn Huff}
%        \ead{kdhuff@illinois.edu}
%  \address[uiuc]{Department of Nuclear, Plasma, and Radiological Engineering,
%        118 Talbot Laboratory, MC 234, Universicy of Illinois at
%        Urbana-Champaign, Urbana, IL 61801}
%
% \end{frontmatter}
\maketitle
\section*{Review General Response}
We would like to thank the reviewers for their detailed assessment of
this paper. Your comments have resulted in changes which certainly improved the 
paper. We have re-run the many simulations in this paper according to the 
suggestions in this work and the results have been updated.


\section*{Reviewer 1}
\begin{questions}
    \question In Table 4 the BWR initial enrichment is now given as 5.8 w/o which seems high to me. Not only does this not match the mean discharge burnup, but it exceeds the 5.0 w/o criticality limit in current fuel fabrication plants. 

    \begin{solution}
    Thank you for your insight. The $5.8 ^{235}U$ value in table 4
    is the initial fissile loading in tonnes. Given the core size
    ($~137$ tonnes), the enrichment is therefore $4.2\%$.
    \end{solution}


    \question I don't understand Figure 9 - there is a blue line captioned "from reprocessed LWR UNF", but the line is hardly showing in the graph. Is this intended?

    \begin{solution}
    The blue line is supposed to be a very thin plane. We intentionally
    plotted this to show that the ASTRID fuel that can be created
    from outstanding French \gls{LWR} \gls{UNF} is very small.
    \end{solution}

    \question Very trivial: Table 12 footnote should read "... are not considered for ..."

    \begin{solution}
    ???
    \end{solution}


\end{questions}
\section*{Reviewer 2}
\begin{questions}
    \question First, the authors refer to the ASTRID reactors as becoming "independent"; I think a more clear term here would be "self-sustaining" in that this fleet is exclusively relying upon plutonium bred from this fleet to sustain core reloadings. (Note that I understand the authors' intent; effectively the ASTRID fleet decouples from the LWR fleet, but it may simply make this point more clearly.)

    \begin{solution}
    That is a very good point, thank you. We have made the changes
    accordingly.
    \end{solution}

    \question Second, in Table 11, I assume the authors mean "Average LWR UNF reprocessed (PER MONTH)" and "(Total) Average UNF reprocessed (PER MONTH)". (Alternatively, these might be expressed as, "Average monthly LWR UNF reprocessing demand" and "Average monthly total UNF reprocessing demand"). I point this out because this can become a bit confusing; the first two entries on this table correspond to absolute values of mass (MT UNF / ASTRID fuel), whereas the second two correspond to RATES (fuel processing demand). The authors make this clear in the text, but I had to go through and re-read after looking at the table to grasp this subtle difference.

    \begin{solution}
    That is a very good point. We made the following changes
    in table 11 for clarity:
    
    Average LWR UNF reprocessed $\rightarrow$ Avg. LWR UNF reprocessed [per month]

    Average UNF reprocessed $\rightarrow$ Avg. UNF reprocessed [per month]

    \end{solution}

    \question Additionally, while a comparison of the average reprocessing demand resulting from LWR life extensions is useful, one other feature I observed is in the peak reprocessing demand (which roughly appears to correlate with the LWR phase-out period and second phase of ASTRID construction). Essentially, it would appear that the peak monthly reprocessing demand is just over 40 MTHM/month, whereas the average demand is well under 30 MTHM. Effectively then, the reprocessing capacity specified in Table 3 doesn't seem to be limiting (unless I am mistaken). The authors appear to assume a doubling of reprocessing capacity post-ASTRID, but this assumption doesn't seem to be required or justified by the physics; if anything, maintaining (or even drawing down) capacity would seem to be in order. I don't think the authors should go back and repeat the study for this, but it is certainly a policy-relevant outcome worth noting.

    \begin{solution}
    ???
    That is a very good point, thank you. Maybe we should clarify
    that the reprocessing demand does fluctuate depending on the
    time (peaks at LWR phaseout and second generation ASTRID
    deployment), and that a large reprocessing capacity is required
    during those periods. The value we have as the average,
    61.3 MTHM per month, is the overall average throughout the
    simulation.
    \end{solution}

    \question Finally, while I know it's nitpicky, you may want to reference the equation on Page 25 by a LaTeX label, rather than relative physical position.

    \begin{solution}
    You are absolutely right. We have made changes accordingly.
    Thank you.
    \end{solution}

    \question Otherwise, I greatly appreciate the effort the authors have put in to clarify working assumptions used in this work, particularly as they pertain to both assumptions used to calculate the depleted fuel recipe as well as assumptions (and the limitations thereof) for the fuel vector supplied to the ASTRID fleet. Some of these assumptions, such as the breeding ratio, while ignoring practical operational considerations are nonetheless instructive as an high-level sensitivity analysis. The additions that the authors have made to caveat and bound these assumptions is quite helpful.

    \begin{solution}
    ???
    [Some words of great appreciation]
    \end{solution}






\end{questions}
%\bibliographystyle{unsrt}
%\bibliography{revise}
\end{document}

  %
  % End of file `elsarticle-template-num.tex'.
