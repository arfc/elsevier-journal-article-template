\section{Introduction}
This paper uses Cyclus, the agent-based simulator \cite{huff_fundamental_2016}, to track
nuclear energy in the EU region. This paper focuses on different scenarios of France,
and how the French legislation and policy can change the nuclear landscape of the European
Union. This paper starts off by doing a generic once through fuel cycle, where all the fuel,
after discharge, goes to a central repository. Next, a reprocessing fuel cycle is evaluated,
where only the French \gls{PWR}s accept \gls{MOX} fuel, while taking all the spent \gls{UOX}
fuel from everywhere in the EU. Finally, a EG01-EG23 transition scenario \cite{wigeland_nuclear_2014},
where a once-through-cycle transitions to a closed cycle with reprocessing and \gls{SFR}s. 

\section{Methodology}
This study mainly utilizes Cyclus, an agent-based simulator, to simulate the fuel cycle
and track material flows in the EU countries. An open-source data is acquired from \gls{IAEA}
\gls{PRIS} as a csv file, listing the country, reactor unit, type, net capacity (MWe), status,
operator, construction date, first criticality date, first grid date, commercial date, shutdown
date (if applicable), and unit capacity factor for 2013. Then only the EU countries are extracted
from the csv file. A python script is written up to generate a Cyclus input file from the csv file,
which lists the individual reactor units as agents. The fuel cycle scenario is defined by modifying
the input file, and the Cyclus simulation is run. Another python script is written to analyze the 
output file, which is in SQL format. Different plots are generated with matplotlib that lists the
output data of the simulation.

\section{Scenario Definition}
The following scenarios are considered: 
\begin{itemize}
	\item Once Through Cycle for all EU reactors
	\item \gls{MOX} reprocessing for French \gls{PWR}s
\end{itemize}
