\section{Introduction}
We used \Cyclus to analyze
the future nuclear inventory in the European Union. \Cyclus is an agent-based extensible
framework for modeling the flow of material through future nuclear cycles.
We calculate the used fuel
inventory in \gls{EU} member states in 2050, and propose a potential collaborative strategy of used fuel
management.
A major focus of this paper is to determine the extent to which France has an incentive
to receive all the \gls{UNF} from \gls{EU} nations to create \gls{MOX}.
The \gls{MOX} created will fuel French transition to a \gls{SFR} fleet
and allow France to avoid building additional \glspl{LWR}.

Past research focuses solely on France, and typically assumes that additional \glspl{LWR},
namely \glspl{EPR} supply the \gls{UNF} required to produce \gls{MOX} \cite{carre_overview_2009, martin_symbiotic_2017, freynet_multiobjective_2016}.
Other recent works implement partitioning and transmutation
in a regional (European) context, with \glspl{ADS} and Gen-IV reactors \cite{fazio_study_2013}.
Little recent work considers synergistic international spent fuel arrangements.
The present work finds that this collaborative strategy can reduce the
need to construct additional \glspl{LWR} in France.



\subsection{\Cyclus}
\Cyclus is an agent-based fuel cycle simulation framework, meaning that
each reactor, reprocessing plant, and fuel fabrication plant is modeled as an agent.
At each timestep (one month),
agents put out their bids for materials (supply and/or demand) and exchange
with one another. A market-like mechanism called the dynamic resource exchange
\cite{gidden_agent-based_2015} governs the exchanges.
Each material item has a quantity, composition, name, and a unique identifier
for output analysis.
A \Cyclus input file contains prototypes, which are fuel cycle facilities with
pre-defined parameters, that are deployed in the simulation as \texttt{facility} agents.
Encapsulating the \texttt{facility} agents are the \texttt{Institution} and \texttt{Region}.
A \texttt{Region} agent holds a set of \texttt{Institution}s.
An \texttt{Institution} agent can deploy or decommission \texttt{facility} agents.
The \texttt{Institution} agent is part of a \texttt{Region} agent,
which can contain multiple \texttt{Institution} agents.


For example, `France' would be a \texttt{Region} agent,
that may contain two \texttt{Institution} agents \glspl{LWR}
and \glspl{SFR}. The \texttt{Institution} agents would then deploy
\glspl{LWR} and \glspl{SFR} agents, respectively, according to a pre-defined deployment
scheme.

